\section{Muhammad Dzihan Al-Banna}
\subsection{Soal 1}
	\lstinputlisting[firstline=9, lastline=21]{src/4/1174095/d_1174095_csv.py}
\subsection{Soal 2}
	\lstinputlisting[firstline=24, lastline=55]{src/4/1174095/d_1174095_csv.py}
\subsection{Soal 3}
	\lstinputlisting[firstline=9, lastline=11]{src/4/1174095/d_1174095_pandas.py}
\subsection{Soal 4}
	\lstinputlisting[firstline=14, lastline=16]{src/4/1174095/d_1174095_pandas.py}
\subsection{Soal 5}
	\lstinputlisting[firstline=19, lastline=20]{src/4/1174095/d_1174095_pandas.py}
\subsection{Soal 6}
	\lstinputlisting[firstline=22, lastline=24]{src/4/1174095/d_1174095_pandas.py}
\subsection{Soal 7}
	\lstinputlisting[firstline=26, lastline=42]{src/4/1174095/d_1174095_pandas.py}
\subsection{Soal 8}
	\lstinputlisting[firstline=8, lastline=10]{src/4/1174095/main_dzihan.py}
\subsection{Soal 9}
	\lstinputlisting[firstline=12, lastline=14]{src/4/1174095/main_dzihan.py}
\subsection{Penaganan Error}
	\lstinputlisting[firstline=8, lastline=11]{src/4/1174095/errdz.py}

\section{Nico Ekklesia Sembiring}
\subsection{Buatlah  fungsi  (file  terpisah/library  dengan  nama  NPMcsv.py)  untuk  membuka file csv dengan lib csv mode list.}
\lstinputlisting[caption = Membuka file CSV dengan lib CSV mode list., firstline=10, lastline=20]{src/4/1174096/praktek/1174096csv.py}

\subsection{Buatlah  fungsi  (file  terpisah/library  dengan  nama  NPMcsv.py)  untuk  membuka file csv dengan lib csv mode dictionary.}
\lstinputlisting[caption =  Membuka file CSV dengan lib CSV mode dictionary., firstline=22, lastline=31]{src/4/1174096/praktek/1174096csv.py}

\subsection{Buatlah fungsi (file terpisah/library dengan nama NPMpandas.py) untuk membuka file csv dengan lib pandas mode list.}
\lstinputlisting[caption =  Membuka file CSV dengan lib Pandas mode list., firstline=9, lastline=11]{src/4/1174096/praktek/1174096pandas.py}

\subsection{Buatlah fungsi (file terpisah/library dengan nama NPMpandas.py) untuk membuka file csv dengan lib pandas mode dictionary.}
\lstinputlisting[caption =  Membuka file CSV dengan lib Pandas mode dictionary., firstline=13, lastline=16]{src/4/1174096/praktek/1174096pandas.py}

\subsection{Buat fungsi baru di NPMpandas.py untuk mengubah format tanggal menjadi standar dataframe.}
\lstinputlisting[caption =  Mengubah format tanggal menjadi standar dataframe., firstline=18, lastline=20]{src/4/1174096/praktek/1174096pandas.py}

\subsection{Buat fungsi baru di NPMpandas.py untuk mengubah index kolom.}
\lstinputlisting[caption =  Mengubah index kolom., firstline=22, lastline=24]{src/4/1174096/praktek/1174096pandas.py}

\subsection{Buat fungsi baru di NPMpandas.py untuk mengubah atribut atau nama kolom.}
\lstinputlisting[caption =  Mengubah atribut atau nama kolom., firstline=26, lastline=30]{src/4/1174096/praktek/1174096pandas.py}

\subsection{Buat program main.py yang menggunakan library NPMcsv.py yang membuat dan membaca file csv.}
\lstinputlisting[caption =  Membuat dan membaca file CSV menggunakan library 1174096pandas., firstline=8, lastline=13]{src/4/1174096/praktek/main.py}

\subsection{Buat program main2.py yang menggunakan library NPMpandas.py yang membuat dan membaca file csv.}
\lstinputlisting[caption = Membuat dan membaca file CSV menggunakan library 1174096pandas., firstline=8, lastline=13]{src/4/1174096/praktek/main2.py}

\subsection{Penanganan Error}
\begin{enumerate}
	\item Tuliskan  peringatan  error  yang  didapat  dari  mengerjakan  praktek  keempat  ini, dan  jelaskan  cara  penanganan  error  tersebut.   dan  Buatlah  satu  fungsi  yang menggunakan gunakan try except untuk menanggulangi error tersebut.
	
	Peringatan error di praktek keempat ini, yaitu:
	\begin{itemize}
		\item Syntax Errors
		Syntax Errors merupakan suatu keadaan pada saat kode python mengalami kesalahan penulisan. Solusi yang dapat dilakukan adalah dengan memperbaiki penulisan kode yang salah.
		
		\item Type Error
		TypeError merupakan exception yang terjadi pada saat dilakukannya eksekusi terhadap suatu operasi atau fungsi dengan type object yang tidak sesuai. Solusi untuk mengatasi error ini adalah dengan mengkoversi varibelnya sesuai dengan tipe data yang akan digunakan.
	\end{itemize}
	
	Fungsi yang menggunakan try except
	\lstinputlisting[caption= Fungsi yang menggunakan try except .,firstline=9, lastline=22]{src/4/1174096/praktek/1174096.py}
\end{enumerate}

\section{Harun Ar-Rasyid}
\subsection{Soal 1}
Isi jawaban soal ke-1

Kalau mau dibikin paragrap \textbf{cukup enter aja}, tidak usah pakai \verb|par| dsb

%\subsection{Soal 2}
%Isi jawaban soal ke-2

%\subsection{Soal 3}
%Isi jawaban soal ke-3

\section{Sri Rahayu}
\subsection{Soal 1}
Isi jawaban soal ke-1

Kalau mau dibikin paragrap \textbf{cukup enter aja}, tidak usah pakai \verb|par| dsb

%\subsection{Soal 2}
%Isi jawaban soal ke-2

%\subsection{Soal 3}
%Isi jawaban soal ke-3

\section{Doli Jonviter}
\subsection{Soal 1}
Isi jawaban soal ke-1

Kalau mau dibikin paragrap \textbf{cukup enter aja}, tidak usah pakai \verb|par| dsb

%\subsection{Soal 2}
%Isi jawaban soal ke-2

%\subsection{Soal 3}
%Isi jawaban soal ke-3

\section{Rahmatul Ridha}
\subsection{Soal 1}
Isi jawaban soal ke-1

Kalau mau dibikin paragrap \textbf{cukup enter aja}, tidak usah pakai \verb|par| dsb

%\subsection{Soal 2}
%Isi jawaban soal ke-2

%\subsection{Soal 3}
%Isi jawaban soal ke-3

\section{Tomy Prawoto}
\subsection{Soal 1}
Isi jawaban soal ke-1

Kalau mau dibikin paragrap \textbf{cukup enter aja}, tidak usah pakai \verb|par| dsb

%\subsection{Soal 2}
%Isi jawaban soal ke-2

%\subsection{Soal 3}
%Isi jawaban soal ke-3
