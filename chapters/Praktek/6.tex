\section{Dezha Aidil Martha}
\subsection{Soal 1}
Buatlah librarri fungsi (file terpisah library dengan nama NPMbar.py) untuk plot dengan jumlah subplot adalah NPM mod 3 tambah 2

\lstinputlisting[firstline=8, lastline=38]{src/6/Praktek/1174025/d1174025_bar.py}

\subsection{Soal 2}
Buatlah librarri fungsi (file terpisah library dengan nama NPMscatter.py) untuk plot dengan jumlah subplot adalah NPM mod 3 tambah  2

\lstinputlisting[firstline=8, lastline=38]{src/6/Praktek/1174025/d1174025_scatter.py}

\subsection{Soal 3}
Buatlah librarri fungsi (file terpisah library dengan nama NPMpie.py) untuk plot dengan jumlah subplot adalah NPM mod 3 tambah 2

\lstinputlisting[firstline=8, lastline=61]{src/6/Praktek/1174025/d1174025_pie.py}

\subsection{Soal 4}
Buatlah librarri fungsi (file terpisah library dengan nama NPMplot.py) untuk plot dengan jumlah subplot adalah NPM mod 3 tambah  2

\lstinputlisting[firstline=8, lastline=38]{src/6/Praktek/1174025/d1174025_plot.py}